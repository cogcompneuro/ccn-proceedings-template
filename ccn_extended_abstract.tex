% -----------------------------------------------------------------------
% CCN Conference - Extended Abstract Track Template
% For papers up to 2 pages (references may extend to 3rd page).
% -----------------------------------------------------------------------
\documentclass{ccn}

\addbibresource{ccn_references.bib}

\title{Formatting instructions for CCN 2025 proceedings submission}
 
\author{{\large \bf Anonymous authors} \\
Double blind review}
%   A Department, 1234 Example Street\\
% A City, State 12345 A country
%   \AND {\large \bf Another Person (AnotherPerson@this.planet.edu)} \\
%   A Department, 1234 Example Street\\
% A City, State 12345 A country}


\begin{document}
\linenumbers

\maketitle


\section{Abstract}
{
\bf
The abstract should be identical to the text version submitted in the web form and should not exceed 300 words. 
CCN has an interdisciplinary audience. Hence a good abstract should
(a) give context about what the problem is and why it matters 
(b) give the contents and explain what was done and what was found
(c) give a clear conclusion including what we learned and how it changes 
the way we think about the universe.
And because Konrad is writing this, he can not avoid shamelessly plugging
his writing guide:
\url{goo.gl/vC8tvf} See you at CCN.
}
\begin{quote}
\small
\textbf{Keywords:} 
add your choice of indexing terms or keywords; kindly use a
semicolon; between each term
\end{quote}

\section{General Formatting Instructions}

The text, tables and figures of a CCN proceedings submission can be no longer than 8
pages. This is including figures, tables, but excluding references.

The text of the paper should be formatted in two columns with an
overall width of 7 inches (17.8 cm) and length of 9.25 inches (23.5
cm), with 0.25 inches between the columns. Leave two line spaces
between the last author listed and the text of the paper. The left
margin should be 0.75 inches and the top margin should be 1 inch.
Use 10~point Modern with 12~point vertical spacing, unless
otherwise specified.

The title should be in 14~point, bold, and centered. The title should
be formatted with initial caps (the first letter of content words
capitalized and the rest lower case). 



Indent the first line of each paragraph by 1/8~inch (except for the
first paragraph of a new section). Do not add extra vertical space
between paragraphs.

\section{Structure}

We recommend a clear structure, typically including an introduction, followed by sections such as methods and results for experimental work (which may be substituted e.g. for theoretical work), and concluded with a discussion.

\section{First Level Headings}

First level headings should be in 12~point, initial caps, bold and
centered. Leave one line space above the heading and 1/4~line space
below the heading.


\subsection{Second Level Headings}

Second level headings should be 11~point, initial caps, bold, and
flush left. Leave one line space above the heading and 1/4~line
space below the heading.


\subsubsection{Third Level Headings}

Third level headings should be 10~point, initial caps, bold, and flush
left. Leave one line space above the heading, but no space after the
heading.


\section{Formalities, Footnotes, and Floats}

Use standard APA citation format. Citations within the text should
include the author's last name and year. If the authors' names are
included in the sentence, place only the year in parentheses, as in
\cite{NewellSimon1972a}
\cite{NewellSimon1972a}, but otherwise place the entire reference in
parentheses with the authors and year separated by a comma
\cite{NewellSimon1972a}. List multiple references alphabetically and
separate them by semicolons
\cite{ChalnickBillman1988a,NewellSimon1972a}. Use the
``et~al.'' construction only after listing all the authors to a
publication in an earlier reference and for citations with four or
more authors.


\subsection{Footnotes}

Indicate footnotes with a number\footnote{Sample of the first
footnote.} in the text. Place the footnotes in 9~point type at the
bottom of the column on which they appear. Precede the footnote block
with a horizontal rule.\footnote{Sample of the second footnote.}


\subsection{Tables}

Number tables consecutively. Place the table number and title (in
10~point) above the table with one line space above the caption and
one line space below it, as in Table~\ref{sample-table}. You may float
tables to the top or bottom of a column, or set wide tables across
both columns.

\begin{table*}[!ht]
\begin{center} 
\caption{Sample table title.} 
\label{sample-table} 
\vskip 0.12in
\begin{tabular}{ll} 
\hline
Error type    &  Example \\
\hline
Take smaller        &   63 - 44 = 21 \\
Always borrow~~~~   &   96 - 42 = 34 \\
0 - N = N           &   70 - 47 = 37 \\
0 - N = 0           &   70 - 47 = 30 \\
\hline
\end{tabular} 
\end{center} 
\end{table*}


\subsection{Figures}

Make sure that the artwork can be printed well (e.g. dark colors) and that 
the figures make understanding the paper easy.
 Number figures sequentially, placing the figure
number and caption, in 10~point, after the figure with one line space
above the caption and one line space below it, as in
Figure~\ref{sample-figure}. If necessary, leave extra white space at
the bottom of the page to avoid splitting the figure and figure
caption. You may float figures to the top or bottom of a column, or
set wide figures across both columns.

\begin{figure}[ht]
\begin{center}
%\fbox{CCN figure}
\fbox{\rule[-.5cm]{0cm}{2cm} \rule[-.5cm]{2cm}{0cm}}
\end{center}
\caption{This is a figure spanning a single column.} 
\label{sample-figure}
\end{figure}

\begin{figure*}[ht]
\begin{center}
%\fbox{CCN figure}
\fbox{\rule[-.5cm]{0cm}{4cm} \rule[-.5cm]{8cm}{0cm}}
\end{center}
\caption{This is a figure spanning both columns.} 
\label{sample-figure}
\end{figure*}


\section{Acknowledgments (camera-ready version only)}

Include acknowledgments \textbf{only} in the final accepted version of the manuscript. Include acknowledgments \textbf{only} in the final accepted version of the manuscript. 


\section{Double-blind review process}

CCN's reviewing process is double-blind, and it is the authors' responsibility to anonymize their submissions. Do not include any identifying information, such as author names, affiliations, or acknowledgments, in the abstract, main text, figures, or metadata. When citing your own work, ensure anonymity to maintain double-blind review standards (e.g., write “In previous work by Author et al. [1]…” instead of “In our previous work [1]...”). If citing a non-anonymous preprint (e.g., from arXiv, social media, or other websites), use anonymized phrasing (e.g., “Author et al. [1] concurrently demonstrate…”). Reviewers are instructed not to actively seek out such preprints, but their discovery does not constitute a conflict of interest. Alternatively, authors may choose not to cite their own non-anonymous preprints, such as those on arXiv. However, prior publications on related topics must be appropriately anonymized when cited.
We encourage including links to code and artifacts in the spirit of open science, but please ensure that the linked material is anonymized; e.g. create a dedicated account to host your material rather than the account of one of the authors. Reviewers are not required to review linked material.

\section{References Instructions}

Follow the APA Publication Manual for citation format, both within the
text and in the reference list, with the following exceptions: (a) do
not cite the page numbers of any book, including chapters in edited
volumes; (b) use the same format for unpublished references as for
published ones. Alphabetize references by the surnames of the authors,
with single author entries preceding multiple author entries. Order
references by the same authors by the year of publication, with the
earliest first.

Use a first level section heading, ``{\bf References}'', as shown
below. Use a hanging indent style, with the first line of the
reference flush against the left margin and subsequent lines indented
by 1/8~inch. Below are example references for a conference paper, book
chapter, journal article, dissertation, book, technical report, and
edited volume, respectively.

\nocite{ChalnickBillman1988a}
\nocite{Feigenbaum1963a}
\nocite{Hill1983a}
\nocite{OhlssonLangley1985a}
% \nocite{Lewis1978a}
\nocite{Matlock2001}
\nocite{NewellSimon1972a}
\nocite{ShragerLangley1990a}

\newpage



%\setlength{\bibleftmargin}{.125in}
%\setlength{\bibindent}{-\bibleftmargin}

\printbibliography


\end{document}
